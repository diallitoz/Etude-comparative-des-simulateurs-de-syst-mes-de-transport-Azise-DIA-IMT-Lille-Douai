Aujourd'hui, le transport occupe une place capitale dans les besoins vitaux des populations au même titre que l'eau, la santé et l'éducation. De plus, face aux questions du réchauffement climatique et de la vie chère, les acteurs politiques se penchent de plus en plus sur les systèmes de transport des personnes afin de proposer des solutions à la fois économique pour le client et écologique.

L'intermodalité qui consiste à utiliser plusieurs modes de transport ou différentes technologies de transport au cours d'un même déplacement, se présente comme une des solutions au problème de mobilité des populations. En effet, la mise en place d'un système de transport moderne, efficace et durable nécessite d'importants moyens financiers. D'après \cite{de2000principes}, pour réduire ces coûts, il faut les hiérarchiser de manière progressive afin de réduire le nombre de lignes et/ou de nœuds du réseau. L'intermodalité permet donc d'accompagner cette réduction des lignes/nœuds en proposant diverses technologies de transport. Par exemple, il est possible de mettre en place des plate-formes de correspondance pour faire un rabattement des personnes des zones peu denses vers des zones plus vaste (principe des gares principales et secondaires).