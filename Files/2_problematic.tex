La mise en place d'un système de transport intermodal nécessite la prise en compte de plusieurs facteurs. Il s'agit notamment :

\begin{itemize}
    \item de la chaîne de transport correspondant aux différents modes de déplacement utilisés (vélo, moto, voiture, bus, tram, train, avion...) ;
    \item de l'offre et son environnement qui correspond à la localisation et au temps (le nombre de stationnement, les itinéraires, les horaires, les correspondances et la durée du trajet...) ;
    \item de l'interface correspondant à l'aménagement, l'information locale et les infrastructures annexes (pôles d'échanges, gares, accessibilité...) ;
    \item du marketing qui correspond à l'information générale, la politique tarifaire, la promotion et la vente ;
    \item de la logistique correspondant à la gestion des flux physiques et informationnels.
\end{itemize}

En général, l'intermodalité est contraignante pour l'usager. Il est donc nécessaire de compenser cet inconvénient par une réelle valeur ajoutée sur le plan de la performance de l'offre de transport sur chacun des points ci-dessus cités afin d'inciter l'usager à changer de mode de transport au cours de son déplacement (voir \cite{wiki:xxx}).

L'atteinte d'une offre de transport performante et adaptée à l'ensemble des usagers nécessite d'une part de disposer des données fiables sur l'ensemble de ces usagers liées notamment à leur habitudes de mobilité, les lieux de fréquentation... D'autre part, il faut construire le réseau de transport (lignes, noeuds, gares, ...) avec toute l'infrastructure adéquate et les matériels roulants (vélo, bus, train...) adéquats. Le réseau de transport doit tenir compte également de certains lieux stratégiques, de son empreinte carbone et surtout des charges de gestion. Ainsi, avant d'entamer la construction du réseau de transport, il est primordial de s'assurer de sont efficacité. Pour y arriver, on pourrait décider par exemple de construire le réseau au fur et à mesure tout en corrigeant les erreurs. Une telle approche n'est pas financièrement et matériellement soutenable et elle demanderait une forte mobilisation humaine sans oublier l'impact sur l'environnement et les populations (l'usager peut perdre confiance au système mis en place). En outre, elle n'est pas renouvelable à souhait. Une solution serait donc de faire une simulation mettant en jeu divers agents (usager, matériel roulant, infrastructure...) avec des comportements dynamiques et prenant en compte les nouvelles technologies du moment notamment l'internet des objets (IoT). On pourra donc agir sur ces agents en fonction des objectifs recherchés et de répéter plusieurs fois les expérimentations afin de choisir les bonnes configurations. Notre travail durant cette thèse consiste donc à proposer ce cadre afin de permettre aux décideurs de disposer d'un outil afin de mettre en oeuvre l'intermodalité dans un contexte d'industrie 4.0.