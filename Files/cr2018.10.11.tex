Compte rendu - réunion de suivi de thèse.
Rapide échange sur l'inscription

Présentation d'Azise sur la base de sa présentation.
Choisir un simulateur: - Présentation de critères utilisés dans des travaux de recherche. * Des critères génériques (outil informatique: communauté, licence, documentation) * Des critères spécifiques (à pondérer): intégration des données, modelé des véhicules * Discution sur l'intérêt de visualisation 3D. - Présentation de critères qui nous sont propres * Langage de programmation. * Interfaces d'accès externe (Systéme de plugging) * Dynamic des agents. - Discution autour d'une thèse réalisée avec GAMA - Étape suivante: Télécharger et tester * Modalié de test à affiner

Source de recherche: * 3 conf. ajouter TRB Transportation Reasearch Board * Voir les journaux/Revue (TR part C) * Voir coté SMA (AAMAS etc.) - Discution sur les références

Administration
Inscription-école doctorale. (Voir avec Emmanuel LEMELIN emmanuel.lemelin@imt-lille-douai.fr)

Attention crédits, prévoir de suivre des formations, MOOC, école d'été...

Grandes phases de la thèse: - Biblio: quelque semaine puis travail au long court - Arbitrer le cas d'étude: Multi-Modalité / Véhicule intelligent. - Viser une première contribution actée par de la publication.

Discution sur la poursuite de carrière: plutôt académique.

Prochiane réunion (visio): 12.11 apm